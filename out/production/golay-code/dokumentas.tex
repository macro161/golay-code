\documentclass[oneside]{VUMIFPSkursinis}
\usepackage{algorithmicx}
\usepackage{algorithm}
\usepackage{algpseudocode}
\usepackage{amsfonts}
\usepackage{float}
\usepackage{amsmath}
\usepackage{bm}
\usepackage{caption}
\usepackage{color}
\usepackage{float}
\usepackage{graphicx}
\usepackage{listings}
\usepackage{subfig}
\usepackage{ltablex}
\usepackage{longtable}
\usepackage{wrapfig}
\usepackage{subfig}
\usepackage{pbox}
\renewcommand{\labelenumii}{\theenumii}
\renewcommand{\theenumii}{\theenumi.\arabic{enumii}.}
\renewcommand{\labelenumiii}{\theenumiii}
\renewcommand{\theenumiii}{\theenumii\arabic{enumiii}.}
\newcolumntype{P}[1]{>{\centering\arraybackslash}p{#1}}
\usepackage[%
	colorlinks=true,
	linkcolor=black
]{hyperref}
\university{Vilniaus universitetas}
\faculty{Matematikos ir informatikos fakultetas}
\department{Programų sistemų katedra}
\papertype{Kodavimo teorija}
\title{Golay kodas}
\titleineng{Golay code}
\status{3 kursas}
\author{Matas Savickis}


\supervisor{Gintaras Skersys, Asist., Dr.}
\date{Vilnius – \the\year}

\bibliography{bibliografija}

\begin{document}
\maketitle

\sectionnonum{Anotacija}
Šio darbo tikslas yra teorines kodavimo teorijos žinias pritaikyti praktiškai įgyvendinant Golay C23 kodą. Darbe įgyvendintas vektoriaus ir teksto užkodavimas, siuntimas nepatikimu kanalu ir dekodavimas. 

\begin{itemize}
	\item{Greta Pyrantaitė - greta.pyrantaite@gmail.com}
	\item{Matas Savickis - savickis.matas@gmail.com}
	\item{Andrius Bentkus - andrius.bentkus@gmail.com}
\end{itemize}

\tableofcontents

\section{Kaip paleisti programa?}
\begin{enumerate}
	\item{1. Atsidaryti komandinę eilutę(CMD)}
	\item{2. Nueiti iki ,,src" aplankalo pateikto zip faile}
	\item{3. Komandinej eilutėj parašyti - javac Main.java}
	\item{4. Komandinėj eilutėj parašyti - java Main}
\end{enumerate}
\section{Pradinių tekstų failai}
\begin{itemize}
	\item{Main.java - Programos įeities taškas}
	\item{Coding.java - Klasė kurioje atliekamas vektoriaus užkodavimas}
	\item{Matrix.java - Klasė kurioje saugomos inicializuotos matricos naudojamos programoje}
	\item{Channel.java - Klasė kuri naudojama iškraipyti vektorių}
	\item{Decoding.java - Klasė kurioje yra įgyvendintas dekodavimas}
	\item{GolayCode.java - Klasė kurioje įgyvendintas funkcionalumų pasirinkimas, vektorių įvedimas, formatavimas ir kviečiamos kitos klasės dirbančios su vektoriumi}
	\item{Utilities.java - Klasė kurioje laikomi pagalbiniai metodai skirti darbui(matricu daugyba, vektoriu atvaizdavimas ir tvarkymas ir t.t.)}
\end{itemize}
\section{Vartotojo sąsaja}

\section{Programiniai sprendimai}
\section{Atlikti eksperimentai}
\section{Naudota literatura}
	Klaidas taisančių kodų teorija Paskaitų konspektai - Gintaras Skersys



\end{document}
